%%%%%%%%%%%%%%%%%%%%%%%%%%%%%%%%%%%%%%%%%%%%%%%%
% 1. document class
%%%%%%%%%%%%%%%%%%%%%%%%%%%%%%%%%%%%%%%%%%%%%%%%
 
\documentclass[a4paper,12pt]{article} % this defines the style of your paper

%%%%%%%%%%%%%%%%%%%%%%%%%%%%%%%%%%%%%%%%%%%%%%%%
% 2. packages
%%%%%%%%%%%%%%%%%%%%%%%%%%%%%%%%%%%%%%%%%%%%%%%%

\usepackage[top = 2.5cm, bottom = 2.5cm, left = 2.5cm, right = 2.5cm]{geometry} 

\usepackage[T1]{fontenc}
\usepackage[utf8]{inputenc}

% the following two packages - multirow and booktabs - are needed to create nice looking tables.
\usepackage{multirow} % multirow is for tables with multiple rows within one cell.
\usepackage{booktabs} % for even nicer tables.

% as we usually want to include some plots (.pdf files) we need a package for that.
\usepackage{graphicx} 

% the default setting of LaTeX is to indent new paragraphs. this is useful for articles. but not really nice for homework problem sets. the following command sets the indent to 0.
\usepackage{setspace}
\setlength{\parindent}{0in}

% package to place figures where you want them.
\usepackage{float}

% the fancyhdr package lets us create nice headers.
\usepackage{fancyhdr}

% citing references will be hyperlinked and blue
\usepackage[colorlinks=true,linkcolor=blue]{hyperref}

\usepackage{amsmath}

\usepackage{tikz}

%%%%%%%%%%%%%%%%%%%%%%%%%%%%%%%%%%%%%%%%%%%%%%%%
% 3. header (and footer)
%%%%%%%%%%%%%%%%%%%%%%%%%%%%%%%%%%%%%%%%%%%%%%%%

% to make our document nice we want a header and number the pages in the footer.

\pagestyle{fancy} % with this command we can customize the header style.

\fancyhf{} % this makes sure we do not have other information in our header or footer.

\lhead{\footnotesize E\&M II}% \lhead puts text in the top left corner. \footnotesize sets our font to a smaller size.

%\rhead works just like \lhead (you can also use \chead)
\rhead{\footnotesize Collins}

% similar commands work for the footer (\lfoot, \cfoot and \rfoot).
% we want to put our page number in the center.
\cfoot{\footnotesize \thepage} 


%%%%%%%%%%%%%%%%%%%%%%%%%%%%%%%%%%%%%%%%%%%%%%%%
% 4. document
%%%%%%%%%%%%%%%%%%%%%%%%%%%%%%%%%%%%%%%%%%%%%%%%

\begin{document}


%%%%%%%%%%%%%%%%%%%%%%%%%%%%%%%%%%%%%%%%%%%%%%%%
%%%%%%%%%%%%%%%%%%%%%%%%%%%%%%%%%%%%%%%%%%%%%%%%


%%%%%%%%%%%%%%%%%%%%%%%%%%%%%%%%%%%%%%%%%%%%%%%%
% title section of the document
%%%%%%%%%%%%%%%%%%%%%%%%%%%%%%%%%%%%%%%%%%%%%%%%

\thispagestyle{empty} % this command disables the header on the first page. 

\begin{tabular}{p{15.5cm}} % this is a simple tabular environment to align your text nicely 
{\large \bf Homework 3}
\\ Collin Collins \\
PHYS 4032\\
17 February 2025 \\
\hline % \hline produces horizontal lines.
\end{tabular} % our tabular environment ends here.

%%%%%%equations that are commonly used%%%%%%%%%%
% \textbf{u}^\prime = \frac{\textbf{u} - \textbf{v}}{1 - \frac{\textbf{uv}}{c^2}}
%
%
%
%
%
%%%%%%%%%%%%%%%%%%%%%%%%%%%%%%%%%%%%%%%%%%%%%%%%

\section*{Problem 1}
Suppose $\mathbf{J}(\mathbf{r})$ is constant in time but $\rho(\mathbf{r}, t)$ is not-conditions that might prevail, for instance, during the charging of a capacitor.\\
\subsection*{Part (a)}
(a) Show that the charge density at any particular point is a linear function of time:
$$
\rho(\mathbf{r}, t)=\rho(\mathbf{r}, 0)+\dot{\rho}(\mathbf{r}, 0) t
$$
where $\dot{\rho}(\mathbf{r}, 0)$ is the time derivative of $\rho$ at $t=0$.\\
\subsubsection*{------------------------------------------Solution-----------------------------------------------}
Seeing $\mathbf{J}(\mathbf{r})$ and $\rho(\mathbf{r}, t)$ makes me think to use the continuity equation:
\begin{equation}
\frac{\partial}{\partial t}\Big[\rho(\mathbf{r}, t)\Big]+\nabla \cdot \mathbf{J}(\mathbf{r}, t)=0
	\label{eq:cont-eq}
\end{equation}
Here, we can use the information given, namely that $\mathbf{J}(\mathbf{r})$ is independent of time:
$$ \frac{\partial}{\partial t}\Big[\rho(\mathbf{r}, t)\Big]+\nabla \cdot \mathbf{J}(\mathbf{r}, t)=0 \quad\rightarrow\quad \frac{\partial}{\partial t}\Big[\rho(\mathbf{r}, t)\Big]+\nabla \cdot \mathbf{J}(\mathbf{r})=0 $$
Rearranging things in a typical way for dealing with differential equations:
$$ \frac{\partial}{\partial t}\Big[\rho(\mathbf{r}, t)\Big]+\nabla \cdot \mathbf{J}(\mathbf{r})=0 \quad\rightarrow\quad \frac{\partial}{\partial t}\Big[\rho(\mathbf{r}, t)\Big]=- \nabla \cdot \mathbf{J}(\mathbf{r})$$
At this point, I see a partial derivative in time that I'd like to remove (to obtain the form of the final result we are asked to prove). Let's do that by integrating both sides with respect to $t'$:
$$ \int_{t'=0}^{t'=t} \frac{\partial}{\partial t'}\Big[\rho(\mathbf{r}, t')\Big] dt' = -\int \nabla \cdot \mathbf{J}(\mathbf{r}) dt' $$
Immediately, my left-hand-side can be simplified with the fundamental theorem of calculus. On the right-hand-side, remember that the integrand is independent of time, and therefore we can treat it as a constant.
$$ \rho(\mathbf{r}, t) - \rho(\mathbf{r}, 0) = -\nabla\cdot\mathbf{J}(\mathbf{r})t \quad\rightarrow\quad \rho(\mathbf{r}, t) = \rho(\mathbf{r}, 0) -\nabla\cdot\mathbf{J}(\mathbf{r})t  $$
We are almost there. The form on the right seems contrived because it is (you've given me the answer and I want whatever I'm doing to be guided by that form). So I've rearranged things to resemble the form we were asked to prove: $\rho(\mathbf{r}, t)=\rho(\mathbf{r}, 0)+\dot{\rho}(\mathbf{r}, 0) t$. Now, we just have to clarify how $\dot{\rho}(\mathbf{r}, 0) = -\nabla\cdot\mathbf{J}(\mathbf{r})$. For that let's remember our earlier result:
$$ \frac{\partial}{\partial t}\Big[\rho(\mathbf{r}, t)\Big]=-\nabla \cdot \mathbf{J}(\mathbf{r}) $$
If this is true for all $t$, then it is certainly true for $t=0$:

$$ \frac{\partial}{\partial t}\Big[\rho(\mathbf{r}, 0)\Big]=-\nabla \cdot \mathbf{J}(\mathbf{r}) \quad\rightarrow\quad \dot{\rho}(\mathbf{r}, 0)=-\nabla \cdot \mathbf{J}(\mathbf{r})  $$
Let's make this substitution in $\rho(\mathbf{r}, t)=\rho(\mathbf{r}, 0)-\nabla \cdot \mathbf{J}(\mathbf{r}) t$.
$$\underline{\boxed{\rho(\mathbf{r}, t)=\rho(\mathbf{r}, 0)-\nabla \cdot \mathbf{J}(\mathbf{r}) t \quad\rightarrow\quad \rho(\mathbf{r}, t)=\rho(\mathbf{r}, 0)+\dot{\rho}(\mathbf{r}, 0) t}} $$




\subsubsection*{-----------------------------------------------------------------------------------------------------}
\subsection*{Part (b)}
This is not an electrostatic or magnetostatic configuration; nevertheless-rather surprisingly-both Coulomb's law (in the form of Eq. 2.8) and the Biot-Savart law (Eq. 5.39) hold, as you can confirm by showing that they satisfy Maxwell's equations. In particular:\\

(b) Show that

$$
\mathbf{B}(\mathbf{r})=\frac{\mu_0}{4 \pi} \int \frac{\mathbf{J}\left(\mathbf{r}'\right) \times \hat{R}}{R^2} d \tau^{\prime}
$$

obeys Ampere's law with Maxwell's displacement current term.
Part b will be considered as extra credit in case you want to turn-in this corrected version.

\subsubsection*{------------------------------------------Solution-----------------------------------------------}
Note that since I don't want to install additional \LaTeX  packages and switch to manual typesetting, I will use $R$ as the separation vector $|R| =: |\mathbf{r} - \mathbf{r}'| $\\

As a reminder, the full Maxwell-Ampère law is
\begin{equation}
\nabla \times \mathbf{B}(\mathbf{r}) = \mu_0\mathbf{J}(\mathbf{r}) + \mu_0\epsilon_0 \frac{\partial}{\partial t} \Big[\mathbf{E}(\mathbf{r}, t)\Big]
	\label{eq:maxwell-ampere-law}
\end{equation}
For this problem, we define the electric field with Coulomb's law:
\begin{equation}
\mathbf{E}(\mathbf{r}, t) = \frac{1}{4\pi\epsilon_0} \int \rho(\mathbf{r}', t)\frac{\mathbf{r} - \mathbf{r}'}{|\mathbf{r} - \mathbf{r}'|^3}d\tau'
	\label{eq:coulomb-law}
\end{equation}
As we showed in the first part, the charge density varies linearly in time and the current density is independent of time. Because $\rho$ is changing, however, we have:
$$ \nabla \cdot \mathbf{J} = -\frac{\partial}{\partial t} \rho $$
To account for this current, we have see the need for the additional term in Eq [\ref{eq:maxwell-ampere-law}]: $\mu_0\epsilon_0\frac{\partial}{\partial t} \Big[\mathbf{E}(\mathbf{r}, t)\Big]$, which provides the missing contribution of current from the static version of Ampère's law.\\

Let's simplify the notation here with
$$ \hat{R} =: \frac{\mathbf{r} - \mathbf{r}'}{|\mathbf{r} - \mathbf{r}'|} \quad \text{and} \quad R =: |\mathbf{r} - \mathbf{r}'| $$
The magnetic field is
$$ \mathbf{B}(\mathbf{r}) = \frac{\mu_0}{4\pi} \int \mathbf{J}(\mathbf{r}')\times \frac{\hat{R}}{R^2} d\tau' $$
Taking the curl, we have
$$ \nabla \times \mathbf{B}(\mathbf{r}) = \frac{\mu_0}{4\pi} \int \nabla \times \left[ \mathbf{J}(\mathbf{r}')\times \frac{\hat{R}}{R^2} \right] d\tau'$$
Let's let (hehe)
$$ \mathbf{a} =: \mathbf{J}(\mathbf{r}') \quad\text{and}\quad \mathbf{A}(\mathbf{r}) =: \frac{\hat{R}}{R^2} $$
We remember the vector identity
$$ \nabla\times\left(\mathbf{a} \times \mathbf{A}(\mathbf{r})\right) = \mathbf{a}\left(\nabla\cdot\mathbf{A}(\mathbf{r})\right) - \left(\mathbf{a}\cdot\nabla\right)\mathbf{A}(\mathbf{r}) $$
Since $\mathbf{a}$ is independent of $\mathbf{r}$, the second term of our vector identity vanishes and we are left with
$$ \nabla\times\left(\mathbf{a} \times \mathbf{A}(\mathbf{r})\right) = \mathbf{a}\left(\nabla\cdot\mathbf{A}(\mathbf{r})\right) \quad\rightarrow\quad \mathbf{J}(\mathbf{r}')\nabla \cdot \frac{\hat{R}}{R^2} $$
From PHYS 4031, we learned that
$$ \nabla \cdot \frac{\hat{R}}{R^2} \quad\rightarrow\quad \nabla \cdot \left[\frac{\mathbf{r} - \mathbf{r}'}{|\mathbf{r} - \mathbf{r}'|^3}\right] = 4\pi\delta^3(\mathbf{r} - \mathbf{r}') $$
Therefore
$$ \mathbf{J}(\mathbf{r}')\nabla \cdot \left[\frac{\mathbf{r} - \mathbf{r}'}{|\mathbf{r} - \mathbf{r}'|^3}\right] \quad\rightarrow\quad 4\pi\mathbf{J}(\mathbf{r}')\delta^3(\mathbf{r} - \mathbf{r}') $$
Making this substitution into integral given in the problem statement,
$$ \boxed{\mathbf{B}(\mathbf{r})=\frac{\mu_0}{4 \pi} \int 4\pi\mathbf{J}(\mathbf{r}')\delta^3(\mathbf{r} - \mathbf{r}') d \tau^{\prime} = \mu_0\mathbf{J}(\mathbf{r})} $$
This is not the end of the problem, as we are not working with a time-independent configuration.\\

Continuing, we consider the time‐dependent electric field term in Eq. \eqref{eq:maxwell-ampere-law}:
\begin{equation}
  \mu_0 \epsilon_0 \frac{\partial}{\partial t}\Big[\mathbf{E}(\mathbf{r},t)\Big]= \mu_0 \epsilon_0 \frac{1}{4\pi\epsilon_0}\int
  \underbrace{\frac{\partial}{\partial t}\Big[\rho(\mathbf{r}',t)\Big]}_{\text{cont. eq.}}
  \frac{\hat{R}}{R^2}d\tau'.
\label{eq:displacement-current-term}
\end{equation}
Since the charge density $\rho$ changes in time but $\mathbf{J}$ and $\rho$ obey the continuity equation, we have
$$
  \frac{\partial \rho(\mathbf{r}',t)}{\partial t}=
  -\nabla' \cdot \mathbf{J}(\mathbf{r}').
$$
Making this substitution,
$$
  \mu_0 \epsilon_0 \frac{1}{4\pi\epsilon_0}
  \int 
  \bigl[\nabla' \cdot \mathbf{J}(\mathbf{r}')\bigr]
  \frac{\hat{R}}{R^2}d\tau'
  \quad\longrightarrow\quad
  -\mu_0 \epsilon_0 \frac{1}{4\pi\epsilon_0}
  \int 
  \bigl[\nabla' \cdot \mathbf{J}(\mathbf{r}')\bigr]
  \frac{\hat{R}}{R^2}d\tau'.
$$

Let's define
$$
  f(\mathbf{r}') := \frac{\hat{R}}{R^2}
  \quad\text{and}\quad
  \mathbf{J}(\mathbf{r}')
  \quad\text{as our vector field.}
$$

Then, using integration by parts (and assuming that any surface term vanishes at infinity):
$$
  \int_{V} \bigl[\nabla' \cdot \mathbf{J}(\mathbf{r}')\bigr]
  f(\mathbf{r}')d\tau'=\underbrace{\oint_{S} f \mathbf{J}\cdot d\mathbf{a}'}_{\text{vanishes}} -\int_{V} \mathbf{J}(\mathbf{r}') \cdot \bigl[\nabla' f(\mathbf{r}')\bigr] d\tau'.
$$
So, 
$$
  -\mu_0 \epsilon_0 \frac{1}{4\pi\epsilon_0}
  \int \nabla' \cdot \mathbf{J}(\mathbf{r}') \frac{\hat{R}}{R^2}
  d\tau'=\mu_0 \epsilon_0 \frac{1}{4\pi\epsilon_0} \int \mathbf{J}(\mathbf{r}') \cdot \nabla' \Bigl[\frac{\hat{R}}{R^2}\Big] d\tau'.
$$

Putting it all together, we see that the total curl of $\mathbf{B}$ is:
$$
  \nabla \times \mathbf{B}(\mathbf{r})=\mu_0\mathbf{J}(\mathbf{r})+\mu_0\epsilon_0\frac{\partial}{\partial t}\Bigl[\mathbf{E}(\mathbf{r},t)\Bigr].
$$
When we include the extra term arising from 
$\partial\rho/\partial t$ (and thus from $\partial\mathbf{E}/\partial t$), the “missing piece” is exactly the displacement current term $\mu_0\epsilon_0\partial\mathbf{E}/\partial t$.  This confirms that a Biot–Savart style $\mathbf{B}$ field \emph{does} satisfy Ampère’s law \emph{including} 
Maxwell’s correction, provided $\rho$ and $\mathbf{J}$ obey the continuity equation.

$$
\underline{\boxed{
   \nabla \times \mathbf{B}(\mathbf{r})=
   \mu_0\mathbf{J}(\mathbf{r})+
   \mu_0\epsilon_0\frac{\partial \mathbf{E}}{\partial t}(\mathbf{r},t)}}
$$

\subsubsection*{-----------------------------------------------------------------------------------------------------}

\section*{Problem 2}
Calculate the force of magnetic attraction between the northern and southern hemispheres of a uniformly charged spinning spherical shell, with radius $R$, angular velocity $\omega$, and surface charge density $\sigma$. [This is the same as Prob. 5.42, but this time use the Maxwell stress tensor and Eq. 8.22.].
\subsubsection*{------------------------------------------Solution-----------------------------------------------}
The Maxwell stress tensor $\overleftrightarrow{T}$ is
$$
T_{ij} = \frac{1}{\mu_0}\Big[B_i B_j - \frac{1}{2}\delta_{ij}B^2\Big]
$$
The force on a volume $V$ bounded by $\partial V$ is
$$
\mathbf{F}=\oint_{\partial V} \overrightarrow{T} \cdot d\mathbf{a}
$$
Since we want the \emph{vertical} ($z$)-component, we only consider this portion of the stress tensor:
$$
(\overrightarrow{T}\cdot d\mathbf{a})_z =\frac{1}{\mu_0} \Big[B_z (\mathbf{B}\cdot d\mathbf{a})-\frac{1}{2}B^2(d a_z)\Big]
$$
  To isolate the force on the upper hemisphere, we take \(\partial V\) to be the combination of
\begin{itemize}
\item A hemisphere of radius $R$ just outside the shell.
\item The disk at $z=0$.
\end{itemize}
We will have to compute the integral over each portion and sum.\\

The net effect inside a uniformly rotating charge distribution is a constant magnetic field. For a thin shell, (I looked this part up)
$$
\mathbf{B}_{\text{inside}}=\frac{2}{3}\mu_0\sigma R\omega\hat{\mathbf{z}}
$$
Outside for ($r > R$), the spinning shell behaves like a magnetic dipole. \\

Its dipole moment is (I also looked this part up because the bulk of this problem seemed to be about actually using the stress tensor)
$$m = \Big(\sigma\omega R\Big) \Big(\frac{4}{3}\pi R^3\Big)=\frac{4}{3}\pi R^4\sigma\omega
$$
So the magnetic field outside is that of a dipole of strength $m$ aligned along $\hat{\mathbf{z}}$ (from a previous Griffiths problem):
$$
\mathbf{B}_{\text{outside}}(r,\theta)=\frac{\mu_0m}{4\pi r^3}\Big[2\cos\theta\hat{\mathbf{r}}+\sin\theta\hat{\boldsymbol{\theta}}
\Big]
$$

With these fields, we can evaluate the stress-tensor surface integrals over the hemisphere (using the $\mathbf{B}$ outside) and over the disk (using $\mathbf{B}$ inside).\\

On the hemisphere just outside $r=R$, the area element is
$$d\mathbf{a}=\big(R^2\sin\theta d\theta d\phi\big)\hat{\mathbf{r}},\quad(\theta \in [0,\tfrac{\pi}{2}],\phi \in [0,2\pi])
$$
The $z$-component of our area element is:
$$
d a_z=(R^2\sin\theta d\theta d\phi)\cos\theta
$$

\noindent
The field on the hemisphere at $r=R$ is 
$$
\mathbf{B}_{\text{outside}}(r=R,\theta)=\frac{\mu_0m}{4\pi R^3}\Big(2\cos\theta\hat{\mathbf{r}}+\sin\theta \hat{\boldsymbol{\theta}}\Big)
$$
The terms in our tensor are:
$$
\boxed{B_z =\mathbf{B}\cdot \hat{\mathbf{z}}=\frac{\mu_0m}{4\pi R^3}\big(3\cos^2\theta -1\big)}
$$
$$
\boxed{\mathbf{B}\cdot d\mathbf{a}=\frac{\mu_0m}{4\pi R^3}
\big(2\cos\theta\big)\big(R^2\sin\theta d\theta d\phi\big)}
$$
$$
\boxed{B^2=\left(\frac{\mu_0m}{4\pi R^3}\right)^2\left(4\cos^2\theta + \sin^2\theta\right)}
$$

The portion of the stress tensor that we are concerned with is 
$$
(\overrightarrow{T} \cdot d\mathbf{a})_z=\frac{1}{\mu_0}\Big[B_z(\mathbf{B}\cdot d\mathbf{a})-\frac12 B^2 \big(d a_z\big)
\Big]
$$
The force is
$$
(F_{\text{hemi}})_z=\oint_{\partial S} \underbrace{\left[\frac{1}{\mu_0}\Big[B_z(\mathbf{B}\cdot d\mathbf{a})-\frac12 B^2 \big(d a_z\big)
\Big]\right]}_{(\overrightarrow{T} \cdot d\mathbf{a})_z}
$$

Using our boxed definitions:
$$
\begin{align}
	(F_{\text{hemi}})_z=\oint_{\partial S}\frac{1}{\mu_0}\Bigg[\underbrace{\frac{\mu_0m}{4\pi R^3}\big(3\cos^2\theta -1\big)}_{B_z}\underbrace{\left(\frac{\mu_0m}{4\pi R^3}
\big(2\cos\theta\big)\big(R^2\sin\theta d\theta d\phi\big)\right)}_{\mathbf{B}\cdot d\mathbf{a}}- \\\ldots \frac12 \underbrace{\left(\frac{\mu_0m}{4\pi R^3}\right)^2\left(4\cos^2\theta + \sin^2\theta\right)}_{B^2} \underbrace{(R^2\sin\theta d\theta d\phi)\cos\theta}_{\big(d a_z\big)}
\Bigg]
\end{align}
$$
When we factor out the differentials and the $\left(\frac{\mu_0m}{4\pi R^3}\right)^2$ term, we have
$$
\begin{align}
	(F_{\text{hemi}})_z=\frac{1}{\mu_0}\left(\frac{\mu_0m}{4\pi R^3}\right)^2\oint_{\partial V}\Bigg[\big(3\cos^2\theta -1\big)\left(2\cos\theta\right)- \\\ldots \frac12 \left(4\cos^2\theta + \sin^2\theta\right)\cos\theta
\Bigg](R^2\sin\theta d\theta d\phi) \quad\rightarrow\quad\ldots
\end{align}
$$
$$ 
(F_{\text{hemi}})_z=\frac{1}{\mu_0}\left(\frac{\mu_0m}{4\pi R^3}\right)^2\oint_{\partial V}\Bigg[\big(6\cos^3\theta -2\cos\theta\big)- \frac12 \left(4\cos^3\theta + \sin^2\theta\cos{\theta}\right)\Bigg](R^2\sin\theta d\theta d\phi)
$$
$$ 
(F_{\text{hemi}})_z=\frac{R^2}{\mu_0}\left(\frac{\mu_0m}{4\pi R^3}\right)^2\underbrace{\int_{\phi}}_{2\pi}\int_{\theta}\Bigg[\big(6\cos^3\theta -2\cos\theta\big)- \frac12 \left(4\cos^3\theta + \sin^2\theta\cos{\theta}\right)\Bigg](\sin\theta d\theta d\phi)
$$
$$ 
(F_{\text{hemi}})_z=\frac{2\pi R^2}{\mu_0}\left(\frac{\mu_0m}{4\pi R^3}\right)^2\int_{\theta=0}^{\theta=\frac{\pi}{2}}\Bigg[\big(6\cos^3\theta -2\cos\theta\big)- \frac12 \left(4\cos^3\theta + \sin^2\theta\cos{\theta}\right)\Bigg](\sin\theta d\theta)
$$
At this point this portion of the problem has been reduced to quadrature. The command I gave Wolfram Mathematica is:
\begin{verbatim}
	Integrate[
	(-2 Cos[a] + 6 Cos[a]^3) Sin[a] + 
	(-4 Cos[a]^3 - Cos[a] Sin[a]^2)/2,
	 {a, 0, Pi/2}
	]
\end{verbatim}
\subsubsection*{-----------------------------------------------------------------------------------------------------}

\end{document}